\documentclass{scrartcl}
\title{Study on the Fibonacci function}
\subtitle{Trying to figure out an $O(log n)$ algorithm}
\author{Otavio Macedo}
\date{December 2011}
\begin{document}
   \maketitle

Let $f(n)$ be the Fibonacci function. Its basic definition is as follows:

\begin{eqnarray*}
f(0) & = & 0\\
f(1) & = & 1\\
f(n) & = & f(n - 1) + f(n - 2), n > 1
\end{eqnarray*}

It seems that the formulation above is just a particular case of the more general definition:

\begin{eqnarray*}
f(0) & = & 0\\
f(1) & = & 1\\
f(2) & = & 1\\
f(n) & = & f(i)f(n - i + 1) + f(i - 1)f(n - i), n > 2, 1 \le i \le n
\end{eqnarray*}

If $n$ is odd, the function can be rewritten as:

$$f(n) = f^2((n+1)/2) + f^2((n+1)/2 - 1)$$

If $n$ is even, there are two other ways to rewrite the function:

\begin{eqnarray*}
f(n) & = & f(n/2).(2f(n/2 - 1) + f(n/2)) \\
      & = & f(n/2).(2f(n/2 + 1) - f(n/2))
\end{eqnarray*}

\end{document}
